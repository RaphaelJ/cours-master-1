\documentclass[a4paper]{article}
\usepackage[utf8]{inputenc}
\usepackage[english]{babel}
\usepackage[babel=true]{csquotes}

\pagestyle{headings}

\title{Computer Network Architectures and Multimedia: Assignement 1}
\author{Julien Nix and Raphaël Javaux}
\date{}

\begin{document}
\maketitle

 \section{Question 1}

   \subsection{TCP behavior}

    \paragraph{}TCP starts sending a packet at a time, waiting for a single ACK.
By increasing the transmission window size, TCP gets its maximum throughput
after around 0.5 second.

   \subsection{n0 to n1 throughput}

    \paragraph{}TCP achives a constant 5.42 Mbps/sec throughput during the last
3 seconds, which is quite close to the maximum bandwidth of the links (6 Mbps).
This confirms our previous observation that TCP reaches its maximum throughput
after half a second.

    \paragraph{}The script can be run with this command :
    \begin{verbatim}
        runhaskell Q1_2.hs < Q1_1.tr
    \end{verbatim}

   \subsection{n0 to n2 falls down}

   \paragraph{}Every packets which were on the n0 to n2 link have been lost.
New packets have been redirected to n2 over the n0 to n1 link (which is already
used by the TCP flow), resulting in the n0's queue overflow.

    \paragraph{}As TCP detected the loss of packets and promptly restarted its
throughput detection algorithm to accommodates itself to the new available
bandwidth of the n0 to n1 link.

   \subsection{Network recovery}

   \paragraph{}The last packet drop happened at 3.23 sec, but TCP only recovers
a ideal throughput around 3.6 sec, after a few iterations of its congestion
detection algorithm.

   \subsection{Packets dropped}

   \paragraph{}

    \paragraph{}The script can be run with this command :
    \begin{verbatim}
        runhaskell Q1_5.hs < Q1_3.tr
    \end{verbatim}

\end{document}